\documentclass[12pt]{article}
\usepackage[utf8]{inputenc}
\usepackage{amsmath}
\usepackage{amssymb}
\usepackage[a4paper, total={6.5in, 9.5in}]{geometry}


\title{Statistical Physics}
\author{Kelvin Ho}
\date{April 2021}

\begin{document}

\maketitle


\section{Introduction}
Summary of Statistical Physics. This is \textbf{not} a course on classical thermodynamics, nor is it strictly a course on statistical thermodynamics.

\section{Classical Laws}

\subsection{Classifications}

\paragraph{Quasi-static systems} are where heat transfer is slow and that thermal equilibrium is maintained with the environment.

\paragraph{Reversible systems}
$dS = 0$ only applies to spontaneous reversible systems.



\subsection{Variables}
\paragraph{Intensive Variables}
Variables which do not scale with the size of the system. Example: Pressure, Temperature, Chemical Potential.

\paragraph{Extensive Variables}
Variables which scale with the size of the system. Example: Entropy, Energy, Volume.

\subsection{Ideal Gas}

The \textbf{Ideal Gas Law} is given by:
\[\boxed{pV = Nk_B T}\]
Where $p$ is pressure, $V$ is volume, $N$ is number of particles, $k_B$ is Boltzmann's Constant, $T$ is temperature.

Thus the total kinetic energy is:

\[ E = \frac12 N m \langle v^2 \rangle\]

The pressure can be shown to be:
\[ pV =\frac13 Nm\langle v^2\rangle\]
Which, when combined:
\[E = \frac32pV\]
Shows that the energy of an Ideal Gas is:

\[ E = \frac32 Nk_B T\]

\subsection{Entropy}
Entropy is simply defined as:

\[ dS = \left(\frac{dQ}{T}\right)_q  \]
Where the underscript $q$ denotes quasi-static system. Rewrite this as:
\[\left(\frac{dQ}{T}\right)_q = \frac{dE + pdV}{T}\]
Performing (reverse) product rule differentiation and make the relevant substitution before integrating:
\[ S(T,V, N) = N k_B\ln{\left(\frac{(k_B T)^\frac32}{\hat{c}N/V}\right)}\]
In which a term related to the number of particles, which does not change in an isolated system, is added. This creates an extensive state variable which is dependent on $T,V,N$.  



\subsection{Four Laws}

\begin{enumerate}
    \item \textbf{Zeroth Law:} If $T_A = T_B$ and $T_B = T_C$, then $T_A =T_C$. Alternatively, heat flow from hotter to cooler systems to achieve equilibrium.
    \item \textbf{First Law:} The energy change of a system is the heat flowing into the system and work done \textbf{on} the system: \[dE = dQ + dW\]
    \item \textbf{Second Law:} Entropy in a system can only be constant (in reversible systems) or increase. \[dS \leq 0\]
    \item \textbf{Third Law:} As temperature goes to absolute zero, entropy approaches a constant (usually zero). In popular saying, you can never reach absolute zero. \[T \to 0^+ \quad S\to S_0\]
\end{enumerate}


\section{Maximising Entropy}

In classical mechanics, systems tend to states which minimises potential energy. In thermodynamics, systems tend to (equilibrium) states which maximises entropy.

\subsection{Fundamental Relation of Thermodynamics}

Using $dW = -pdV$ and $dQ = TdS$, one can write:

\[\boxed{dE = TdS - pdV + \mu dN}\]

This can be expressed as:

\[
dE = \underset{=T}{\left(\frac{\partial E}{\partial S}\right)}dS

\]
\subsection{Legendre Transformation}




\subsection{Entropy Minimisation}
The system lives in an environment in which heat can be exchanged. However, the environment's temperature remains constant. If $dQ$ flows into the system, then the environment experiences $-dQ$, so the entropy change of the environment is:
\[ dS_\text{env} = \frac{-dQ}{T_\text{env}}\]

The total entropy is:
\begin{align*}
    dS_\text{total} &= dS_\text{env} + dS\\
    &= \frac{-dQ}{T_\text{env}}+0\\
\end{align*}


\end{document}