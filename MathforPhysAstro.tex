\documentclass[12pt]{article}
\usepackage[utf8]{inputenc}
\usepackage{amsmath}
\usepackage{amssymb}
\usepackage[a4paper, total={6.5in, 9.5in}]{geometry}
\usepackage{enumerate}
\usepackage[T1]{fontenc}

\usepackage{palatino}


\title{Mathematics For Physics and Astronomy}
\author{Kelvin}

\begin{document}

\maketitle



\section{Complex Analysis}


A complex function is given by:
\[f(z) = u(x,y) + iv(x,y)\]

Its derivative would similarly be defined as:\[ \frac{df(z)}{dz} = \lim_{\Delta z\to z_0}\frac{f(z)}{\Delta z}\]
Which has to exist and has the same value regardless of the direction of approach.



\subsection{Cauchy-Riemann Conditions}
\begin{align*}
    \frac{\partial u}{\partial x} &= \frac{\partial v}{\partial y}\\
    \frac{\partial v}{\partial x} &= -\frac{\partial u}{\partial y}
\end{align*}

If $u$ and $v$ and their derivatives are continuous and satisfy the Cauchy-Riemann conditions, $f(z)$ is analytic in the region. 

\subsection{Expansion}
If $z_0$ is in a region in which $f(z)$ is analytical, $f(z_0)$ can be expanded as a Taylor Series which has a circle of convergence (convergent limit) to the nearest singular point.

\subsection{Harmonic}
\begin{enumerate}
    \item If $f(z)$ is analytical in a region, $u$ and $v$ are harmonic and satisfies Laplace's equation
    \item (Corollary?) Any $u$ or $v$ which satisfies Laplace's Equation corresponds to the real or imaginary part of a complex function. 
\end{enumerate}

\subsection{Cauchy's Theorem}
In a region enclosed by a simple connected curve $C$, if $f(z)$ is analytic in and on the $C$, then:

\[ \oint_C f(z) dz = 0\]

This can be proved using Green's Theorem and the Cauchy-Riemann Conditions. Green's theorem is given by:
\[\iint \frac{\partial Q}{\partial y} + \frac{\partial P}{\partial x} dy\,dx = \oint Qdx - \oint P dy\]

\subsection{Cauchy's Integral Formula}

For $z=a$ inside a region in which $f(z)$ is analytic:

\[ f(a) = \frac{1}{2\pi i} \oint_C \frac{f(z)}{z-a} dz\]

\textsc{Note}: If $a$ lies outside the region, $\frac{f(z)}{z-a}$ is analytic in $C$, and would be equal to zero by Cauchy's Theorem.

\subsection{Laurent's Theorem}

Let $C_1$ and $C_2$ be concentric circles with the same center $z_0$. If $f(z)$ is analytic in the region $R$ between the circles (outside $C_1$ but inside $C_2$), then $f(z)$ can be expanded in a series of the form:
\[ f(z) = a_0 + a_1(z-z_0) + a_2(z-z_0)^2+a_3(z-z_0)^3 +...+\frac{b_1}{(z-z_0)}+\frac{b_2}{(z-z_0)^2}+...\] 
Which is convergent in $R$. This series is called a \textit{Laurent Series}, with the $b$ series being the \textit{principal part}.

\subsubsection{Coefficients}
While the coefficients can be found by the formulas:
\[
a_n = \frac{1}{2\pi i} \oint_C \frac{f(z)dz}{(z-z_0)^{n+1}} \quad\text{and}\quad b_n = \frac{1}{2\pi i} \oint_C \frac{f(z)dz}{(z-z_0)^{-n+1}}
\]
However, a simpler way is often to expand a function in partial fractions and write it as a power series. 

\paragraph{Definitions}
\begin{enumerate}
    \item If there is no $b$ series (i.e. all $b$ are zero, $f(z)$ is analytic at $z=z_0$ and $z_0$ is a \textit{regular point}
    \item If $b$ terminates after $b_n$, then $f(z)$ has a \textit{pole of order $n$}. If $n=1$, $f(z)$ has a simple pole.
    \item If the $b$ series is infinite, then it has an \textit{essential singularity} in $1=z_0$.
    \item $b_1$ is the \textit{residue} of $f(z)$ at $z=z_0$.
\end{enumerate}

\subsection{Finding Residues}
There are a few methods to find residues:

\paragraph{Laurent Series}
Simply expand $f(z)$ as a Laurent Series.

\paragraph{Simple Pole}
If $f(z)$ has a simple pole, multiply by $(z-z_0)$ and evaluate it at $z=z_0$:

\[R(z_0) =\lim_{z\to z_0} (z-z_0)f(z)\]

\paragraph{Function is one divided by another}

For $f(z) = g(z)/h(z)$ and $g(z_0) = \text{finite constant}$ but $\neq 0$ and $h(z_0) = 0$ by L'Hopital's rule:
\[ R(z_0) = \frac{g(z_0)}{h'(z_0)}\]

\paragraph{Multiple Poles}

\[R(z_0) = \frac{1}{(m-1)!}\frac{d^{(m-1)}}{dx^{(m-1)}}(z-z_0)^mf(z)\]


\subsection{Residue Theorem}
For an integral around $C$ in the counterclockwise direction:

\[\oint_Cf(z)dz = 2\pi i \sum \text{residues in } C\]

\subsection{Applications: Evaluation of Definite Integrals}

\subsubsection{Change of Variables}
In an integral with rational function of sine and cosine between $[0,2\pi]$ or (depending on whether the function is odd or even) $[0,\pi]$, do a change of variable $z = e^{i\theta}$. The integral can subsequently be found with the Residue theorem.

\subsubsection{Stretching (Jordan's Lemma)}
An integral of the form:
\[ \int^\infty_{-\infty}\frac{P(x)}{Q(x)}dx\]
Where $Q(x)$ is at least two degrees greater than $P(x)$ and if $Q(z)$ has no real zeroes (\textit{holes} in the $x$ axis) can be evaluated. If $P(x)/Q(x)$ is even, then the integral can also be evaluated for $[0,\infty]$.

This is because this can be rewritten as:
\[\oint_C\frac{P(x)}{Q(x)} = \int\rho_{-\rho}\frac{P(x)}{Q(x)}dx + \int_0^\pi \frac{P(e^{i\theta})}{Q(e^{i\theta})}d\theta\]
Whereby stretching $\rho\to\infty$, the last integral dies, and LHS is evaluated by Residue theorem. As such:
\[\oint_C \frac{P(X)}{Q(x)} = \int^\infty_{-\infty}\frac{P(x)}{Q(x)}\]


\section{Calculus of Variations}

\subsection{Euler-Lagrange Equation}
For a functional:
\[ \int L(x,y,y') dx\]

Then the Euler-Lagrange equation gives the function which minimises the integral:

\[ \frac{d}{dx}\left(\frac{\partial L}{\partial y'} \right) -\frac{\partial L}{\partial y} = 0\]

In classical mechanics, a Lagrangian can be written as:
\[ L = T -V\]
Which $T$ is the kinetic energy and $V$ is the potential energy. Euler-Lagrange equation would give the equation of motion for the system.

\subsection{Beltrami's Identity}
For a special case of a functional in the form $F(y,y')$:
\[ F - y'\frac{\partial F}{\partial y'} = C\]
Where $C$ is an unknown constant. This is proven by finding the derivative of \[ \frac{d}{dx}\left(F - y'\frac{\partial F}{\partial y'}\right)\] and recovering EL's equation. 


\section{Group Theory}

\paragraph{Definition}
A group ${G,*}$ consists of a set $G$ and a binary operation $*$. Writing $a*b$ means do $b$ first before doing $a$. This is a mapping of $*:G\times G\to G$
\subsection{Group Axioms}
\begin{description}
\item [Closure] For $f,g \in G$, $f*g \in G$.
\item [Identity] For every element in $G$, $e * f = f* e = f$
\item [Inverse] For every element in $G$, $f^{-1} \in G$ and $f*f^{-1} = f^{-1}*f =e$
\item [Associativity] For $f,g,h\in G$, \(
(f * g)*h = f*(g*h)\)
\end{description}

\paragraph{Abelian Group} If a group commutes (order does not matter), it is \textit{abelian}.

\paragraph{Order} The order of a (finite) group is defined by the number of elements in the group $|G|$. 

\subsection{Example of Groups}

\subsubsection{Cyclic Groups}
This is rotating a $n$ sided polygon. Each operation rotates it by $2\pi/n$ radians. Thus:

\[ C_n = \{e,r,r^2,r^3,...,r^{n-1}\}\]

\subsubsection{Dihedral Groups}
This group defines the reflections and rotations of a $n$ sided polygon. $|D_n| = 2n$.
\subsubsection{Permutation Groups}
A permutation group simply permutes elements in a set. It can be written in a few different notations:
\[ \sigma = 
\begin{pmatrix}
1 & 2 & 3 & 4\\
3 & 1 & 2 & 4
\end{pmatrix}\]
Where the second row describes where to move the elements to.

Another notation is:
\[ (132) \]
Which describes the same thing:
\[ 1 \to 3 \qquad 3 \to 2 \qquad 2\to 1\]
\subsection{Multiplication Table}
This table defines the results of operating with the group elements (I think it is also called Cayley Tables). For example, the table for $C_3$ is:

\begin{table}[h]
    \centering
    \begin{tabular}{c|ccc}
         $C_3$ & $e$ & $r$ & $r^2$  \\
         \hline
         $e$ & $e$ & $r$ & $r^2$\\
         $r$ & $r$ & $r^2$ & $e$\\
         $r^2$& $r^2$ & $e$ & $r$
    \end{tabular}
    \label{tab:c3}
\end{table}

\paragraph{Rearrangement Theorem}
Each element must only appear in each column or row once. (Like Sudoku.)

\subsection{Isomorphic Groups}
Groups are isomorphic if they have the same multiplication table. If two groups are isomorphic:
\[ F \cong G\]



\subsection{Subgroups}
This is where a smaller group exists within a larger group with the same operator:
\[ H \subseteq G\]
If this is the case, only the \textbf{inverse} and \textbf{closure} axiom have to be checked.

\paragraph{Order of an element}
The order of an element is defined to as the number of times it has to operate on itself to become the identity element:
\[ g^d = e\]

This property can be used to generate cyclic subgroup of a group.

\subsection{Homomorphism}

Let $(G,*)$ and $(H,\cdot)$ be two groups. Let $f$ be a function which maps elements in $G$ to $H$, so $f: G\to H$. If $G$ and $H$ is homomorphic:
\[ f(x*y) = f(x) \cdot f(y)\]

\subsection{Representation}
A representation is the mapping $f: G \to GL_n(\mathbb{C})$, where $GL_n(\mathbb{C})$ is the \textit{general linear group}.

\subsubsection{Representation of $D_3$}
The dihedral group ($D_n$) has a 2D representation because they represent linear transformations in 2D space. For $D_3$, let a triangle has vertices at $1\to (0,1), 2\to(1,0), 3\to(-1,0)$

Any clockwise rotation is the transformation given by the matrix:
\[
\begin{pmatrix}
\cos\theta & \sin\theta\\
-\sin\theta & \cos\theta
\end{pmatrix}
\]

Then, $r$ corresponds to a clockwise rotation of $120^\circ$ and $r^2$ corresponds to a clockwise rotation of $240^\circ$. The representation is given by:
\[
\rho(r) = 
\begin{pmatrix}
-1/2 &{\sqrt3}/{2}\\
-{\sqrt3}/{2} & -1/2
\end{pmatrix}
\qquad
\rho(r^2) = 
\begin{pmatrix}
-1/2 &-{\sqrt3}/{2}\\
{\sqrt3}/{2} & -1/2
\end{pmatrix}
\]
A reflection on the $y$ axis is given by:
\[\rho(a) = 
\begin{pmatrix}
-1 & 0\\
0 & 1
\end{pmatrix}\]
It is known that the representation is homomorphic. Reflection along other axis is the same as reflection followed by rotation: $b = ar$. By the definition of homomorphism, $\rho(b)=\rho(a)\rho(r)$:
\begin{align*}
    \rho(a)\rho(r) &= 
\begin{pmatrix}
-1 & 0\\
0 & 1
\end{pmatrix}
\begin{pmatrix}
-1/2 &{\sqrt3}/{2}\\
-{\sqrt3}/{2} & -1/2
\end{pmatrix} 
=\begin{pmatrix}
1/2 &-{\sqrt3}/{2}\\
-{\sqrt3}/{2} & -1/2
\end{pmatrix}\\
\rho(a)\rho(r^2) (= \rho(r)\rho(a)) &=
\begin{pmatrix}
-1 & 0\\
0 & 1
\end{pmatrix}\begin{pmatrix}
-1/2 &-{\sqrt3}/{2}\\
{\sqrt3}/{2} & -1/2
\end{pmatrix}
=
\begin{pmatrix}
1/2 &{\sqrt3}/{2}\\
{\sqrt3}/{2} & -1/2
\end{pmatrix}
\end{align*}

\subsubsection{Determinant Representation}
This can be taken one step further. The determinant representation is defined to be the determinant of the `natural' representation. Let $G$ be a group, then its determinant representation is:
\[ f(g) = \det (g)\]
For the natural representation above:
\[
\begin{matrix}
f(e) = 1 & f(r) = 1 & f(r^2) = 1\\
f(a) = -1 & f(b) = -1 & f(c) = -1
\end{matrix}
\]
Which distinguishes between rotations and reflections.

\subsubsection{Equivalent Representations}
Suppose another representation is defined as:
\[\gamma(g) = P^{-1}\rho(g)P\]
Then:
\[\gamma(gh) = P^{-1}\rho(gh)P = P^{-1}\rho(g)\rho(h)P = P^{-1}\rho(g)PP^{-1}\rho(h)P = \gamma(g)\gamma(h)\]

Thus in general:
\paragraph{Definition}
Two representations are equivalent if they can be expressed as:
\[\rho(h) = P^{-1}\sigma(h)P\]
Two representations are equivalent if they have the same dimensions.

For example, $\rho$ and $\theta$ cannot be equivalent representations of group $G$ if either one of them are equal to the identity matrix $I_n$.

Another example is 1 dimensional representations are only equivalent iff they are equal (as they commute).

\subsubsection{Reducible Representations}
A representation is reducible if it is in a block diagonal matrix. 

Determining whether representations are reducible can be performed by using equivalent representations.

\subsection{Characters}
The \textit{character} of a representation is the trace of the representation. If we call the character of representation $\rho$ as $\chi_\rho$, then $\chi_\rho:G\to \mathbb{C}$:
\[ \chi_\rho(g) = \text{Tr}\rho(g)\]

\subsubsection{Conjugate}
Two element \(x,y \in G\) are conjugate if there is a \(g \in G\) where \(y = g^{-1}x g\).

\paragraph{Lemma} Conjugates have equivalent characters\footnote{Characters are a function of class.}.

A conjugacy class is given by:
\[ [x] = \{y \in G : x \,\text{and} \,y \text{ are conjugates}\} \]

This is useful because \textit{the number of irreducible representations are equal to the number of conjugate classes}

The group $D_3$ has three conjugacy classes:
\[ [e] = \{e\} \qquad [r] = \{r,r^2\} \qquad [a]=[b]=[c] = \{a,b,c\}\]

Let $\rho$ be a reducible representation.
\begin{itemize}
    \item The irreducible representations of $G$ is $r$
    \item The conjugacy classes of $G$ is $c$
\end{itemize}
Then:
\begin{itemize}
    \item $n_r$ refers to the number of copies of irreducible representations in $\rho$
    \item $n_c$ is the number of conjugacy classes in $G$
\end{itemize}

Then:
\[|G|n_r = \sum_c n_c \overline{\chi_r(c)}\chi_p(c)\]
Where $\overline{\chi_r(c)}$ is the complex conjugate.\ref{tab:c3}
\end{document}