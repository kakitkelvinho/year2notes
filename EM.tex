\documentclass[12pt]{article}
\usepackage[utf8]{inputenc}
\usepackage{amsmath}
\usepackage{amssymb}
\usepackage[a4paper, total={6.5in, 9.5in}]{geometry}



\title{Electrodynamics}
\author{Kelvin Ho}
\date{April 2021}

\begin{document}
\maketitle

\section{Introduction}
I am going to keep this brief because Alex already made a really good formula sheet, there are multitudes of these in the back of textbooks, such as the one on Griffiths, and there are some on moodle too. 



\section{Electric Fields in Matter}

\subsection{Polarization}

\[ p = \alpha E\]

Vector points in the same direction as the E field. Describes how an atom is polarized due to how the electrons and protons act under the field.

\subsection{Dipole Molecules}


Dipole moment is defined to be:

\[\mathbf{p}= \int r' \rho(r) d\tau\]

In the case of discrete collection of charges:
\[ \mathbf{p} = \sum q_i r_i \]

In the case of a dipole:
\[ \mathbf{p} = +q r_+ + (-q)r_- = q(r_+-r_-) = q\mathbf{d}\]

The torque of a dipole in an electric field is:
\[ \mathbf{N} = \mathbf{p} \times \mathbf{E}\]
The potential energy of an ideal dipole in an electric field is:

\begin{align*}
    V = qV(r+d) - q(Vr)
\end{align*}









\end{document}