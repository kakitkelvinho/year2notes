\documentclass[12pt]{article}
\usepackage[utf8]{inputenc}
\usepackage{amsmath}
\usepackage{amssymb}
\usepackage[a4paper, total={6.5in, 9.5in}]{geometry}
\usepackage{enumerate}



\title{Atom and Molecular Physics}
\author{Kelvin Ho}
\date{April 2021}

\begin{document}

\maketitle

\section{Introduction}
This is a continuation of the first Quantum Mechanics course. Note: My personal opinion but as far as this course goes it feels more like chemistry than physics. Maybe my fundamentals are bad, which is why I feel this way, but seriously, I am really bad at Chemistry.

\section{One-electron Atoms}

\subsection{Hydrogen}

Solving the Schrodinger Equation for the Hydrogen Atom is to write the potential of the system as the Colomb potential between the electron and the ion core:

\[V(r) = \frac{-Ze^2}{4\pi\epsilon_0r}\]

Then, the energy levels are:

\[ E_n = -\frac{m_eZ^2e^4}{32\pi^2\epsilon_0^2\hbar^2n^2}\]

The energy emitted when an electron jumps is the change between the energies of the system:

\[\Delta E_n = -\frac{m_eZ^2e^4}{32\pi^2\epsilon_0^2\hbar^2}\left(\frac{1}{n_1^2} - \frac{1}{n_2^2}\right)\]

Rydberg empirically discovered that:

\[\frac{1}{\lambda_{if}} = R\left(\frac{1}{n_f^2} - \frac{1}{n_i^2}\right)\]

Compared with the first equation and Planck's formula $E = \frac{hc}{\lambda}$:

\[ R = \frac{m_eZ^2e^4}{64\pi^2\epsilon_0^2\hbar^3c}\]


\subsection{Finite Mass Core}
The reduced mass is:

\[\mu = \frac{1}{m_p} + \frac{1}{m_e} = \frac{m_e m_p}{m_e + m_p}\]

For example, for the adjusted Rydberg constant of Hydrogen:

\[ R_H = R_\infty \frac{\mu_H}{m_e}\]

In general, for a one electron atom of mass $M$ and ion core charge given by $Ze$:

\[R_M = R_\infty\frac{\mu_M}{m_e}Z^2\]

\subsection{Wavefunction}
The wavefunction is solved in the previous quantum module. They can be represented as:
\[ \psi = R(r)Y^m_l(\theta,\phi)\]

\subsubsection{Angular Part}
The nature of the wavefunction indicates that the electron's angular momentum is quantised and relies on $l$ and $m$, which is the \textit{orbital angular momentum quantum number} and the \textit{magnetic/azimuthal quantum number}. They can vary in integer levels:
\begin{align*}
    l &= 0, 1, 2,..., n - 1\\
    m &= -l, -l+1,...,l-1,l
\end{align*}

The $l$ values are also labelled with alphabets $s,p,d,f,g,h,...$.

\paragraph{Parity}
This is defined to be:
\[\hat{P} f(\vec{r}) \equiv f(-\vec{r})\]
Even parity:
\[\hat{P} f(\vec{r}) \equiv f(-\vec{r}) =  f(\vec{r})\]
Odd Parity:
\[\hat{P} f(\vec{r}) \equiv f(-\vec{r}) =  -f(\vec{r})\]

When applied to spherical harmonic function, this is multiplying the wavefunction with $(-1)^m$.

\subsubsection{Radial Part}

Solving for the radial wavefunction indicates that this only depends on $n$ and $l$.

\subsubsection{Degeneracy}

The energy level, shown by the Rydberg formula, is only dependent on the principle quantum number $n$. Given that there are $n-1$ $l$ for each $n$ and $2l+1$ $m$ for each $l$, there are a total of:

\[\sum^{n-1}_{l=0} (2l+1) = n^2\] degenerate states.

Reasons for degeneracy:$\frac1r$ Potential is only dependent on magnitude of $r$ but not on the (angular) direction. 

\subsection{Spin}

Arises from the relativistic treatment of electron motion. Is very similar to the analysis of angular momentum. Instead of $l$ and $m$, there are $s$ and $m_s$, representing the spin vector and the projection of the spin vector in the $z$ dimension. The full wavefunction is adjusted by introducing the spin dependence:

\[\psi = R_n(r)Y^l_m(\theta, \psi) \chi^s_{m_s}\]

\subsection{Transitions}

When a photon is absorbed, the electron `jumps'. The selection rules are:

\begin{itemize}
    \item $\Delta n$: Any value
    \item $\Delta l = \pm 1$: TO conserve angular momentum
    \item $\Delta m = 0, \pm 1$
\end{itemize}    
    
    
\section{Many Electron Atom}

This is more difficult to solve because the potential would hence be dependent upon the interaction between the electrons. 

\subsection{Independent Particle Model}
First ignore the electron electron repulsion. Solve for each and every electron and solve for it. Then correct for it.

\subsection{Central Field Approximation}
Add another potential term:
\[ V_c(r_i) = \langle \frac{e^2}{4\pi\epsilon_0 r_{ij}}\rangle_{j\neq i}\]
Which is basically the Colomb potential between different electrons. Then, solve $N$ one-electron Schrodinger equation, and the total energy is the sum of the energy from each of the solutions. 

The potential is not dependent on $\frac{1}{r}$ anymore.

This method is best applied to atoms or ions with a single electron occupying an outer orbital:
\begin{itemize}
    \item Alkali Metal Atoms in Group I
    \item Slightly Ionised Alkaline Earth Metals in Group II (After losing one electron in the S orbital)
    \item Rydberg atoms: These have one electron in the highly excited state (high value of n)
\end{itemize}

\paragraph{Screening}
The inner electrons \textit{screen} the charge of the nuclues:

When $r_i \to 0$:
\[ -\frac{Ze^2}{4\pi\epsilon_0 r_i} + V_c(r_i) \implies -\frac{Ze^2}{4\pi\epsilon_0 r_i}\]

When $r_i \to \infty$:
\[ -\frac{Ze^2}{4\pi\epsilon_0 r_i} + V_c(r_i) \implies -\frac{e^2}{4\pi\epsilon_0 r_i}\]

\subsection{Quantum Defects}

Solving the Schrodinger Equation leads to an adjustment to the eigenenergies:

\[\frac{E_{nl}}{hc} = -\frac{R_M}{(n-\Delta_{nl})^2}\]

Where $\Delta_{nl}$ is the quantum defect and $(n-\Delta_{nl})$ is the effective principal quantum number. The adjustment to the Rydberg constant uses a $Z$ that is the effective combined charge of the nucleus and the core electrons. See page 43 for a table listing of some quantum defect values.

\subsection{Indistinguishable particles}

If two particles are indistinguishable by measurement, then:

\[ |\psi(1,2)|^2 = |\psi(2,1)|^2\]

Which means:
\[ \psi(1,2) = +\psi(2,1)\]
Which is known as symmetric w.r.t. exchange.

Or:
\[ \psi(1,2) = -\psi(2,1)\]
Which is known as antisymmetric w.r.t. exchange.

\subsection{Pauli Principle}

\begin{itemize}
    \item Fermions: Antisymmetric
    \item Bosons: Symmetric
\end{itemize}

Thus, no two identical fermions can occupy the same state.

\subsection{Spin Wavefunctions}

An individual electron can have:
\begin{align*}
    s &= \frac{1}{2}\\
    m_s &= \pm\frac12\\
\end{align*}
Depending on whether it is aligned with the positive or negative $z$ axis, otherwise known as spin up or spin down.

As a result:

\[\vec{S} = \vec{s_1} + \vec{s_2}\]
Thus, the total spin quantum number:
\[S = |s_1-s_2|,...,|s_1+s_2|\]
In general, the total spin vector, total spin number and spin projection are: $\vec{S}, S, M_s$.

\subsection{Triplet and Singlet States}

When $S=1$, $M_S = -1,0,1$. This is known as the \textit{triplet} spin state. In general, exchanges in the triplet state are \textit{symmetric} w.r.t. to exchange. This is why the symmetric antiparallel state is chosen when $M_S = 0$.
\begin{itemize}
    \item $M_S = +1$: Then, $\chi^T = \upuparrows = \alpha(1)\alpha(2)$
    \item $M_S = -1$: Then, $\chi^T = \downdownarrows = \beta(1)\beta(2)$
    \item $M_S=0$: Then, $\chi^T = \frac{1}{\sqrt{2}}[\uparrow\downarrow + \downarrow\uparrow] = \frac{1}{\sqrt{2}}[\alpha(1)\beta(2)+\beta(1)\alpha(1)$
\end{itemize}

When $S = 0$, $M_S = 0$. This is the \textit{singlet} spin state, and the antiparallel state is \textit{antisymmetric}:$\chi^T = \frac{1}{\sqrt{2}}[\uparrow\downarrow - \downarrow\uparrow] = \frac{1}{\sqrt{2}}[\alpha(1)\beta(2)-\beta(1)\alpha(1)$

This can be generalized to the He atom wavefunctions (see pg 49-50).

\paragraph{Spin Multiplicity}

In general, each spin state has $2S+1$ associated values of $M_S$. Thus, $S=1 \implies 2S+1 = 3$ which is triplet states.

\subsection{Splitting and Exchange}

The excited state can be either in singlet or triplet state. 
\begin{itemize}
    \item \textit{Singlet}: Antisymmetric spin, so symmetric spatial. There is nonzero probability for $r_1=r_2$.
    \item \textit{Triplet}: Vice versa. Thus, as the wavefunction cannot be zero, $r_1\neq r_2$.
\end{itemize}

If the electrons can be closed to each other, they can screen the $Z=+2$ charge of the nucleus, resulting in a weakly bound state, which is higher in energy. Vice versa, being strongly bound results in a lower energy (something like potential energy?), so there is a splitting.

\subsection{Formal Analysis}
From here on to page 56 there is a formal analysis with integrals and what not. Seems to be too hard to accessed.

\subsection{Electronic configurations}

Basically chemistry stuff:

\begin{itemize}
    \item $n$ refers to the shell
    \item $l$ refers to the orbitals
    \item $n=l$ means same subshell
    \item Each shell can hold $2n^2$ electrons.
    \item Electron in open shells (not full) are optically active
\end{itemize}
    
\subsection{Term Symbols}
Before I continue, note that this is quite rushed and the course itself is not very rigorous nor detailed so I am going to play it fast and loose with the notations.

The many electron Hamiltonian commutes with $\hat{L}^2$ and $\hat{S}^2$. Thus, Physicists decided it would be wise to have a notation with these values. Big $L$ and $S$ are the total of their smaller alphabets components:

\begin{align*}
    \vec{L} &= \vec{l_1}+\vec{l_2}\\
    L &= |l_1 - l_2|,...,|l_1+l_2| \qquad \text{in steps of 1}\\
    \vec{S} &= \vec{s_1}+\vec{s_2}\\
    L &= |s_1 - s_2|,...,|s_1+s_2| \qquad \text{in steps of 1}\\
\end{align*}


\paragraph{Parity revisited}
For equivalent electrons (living in the same orbital), the parity of the spherical harmonic function reveals the symmetry of the spatial wavefunction. Luckily this is just given by $(-1)^L$, with $+1$ being symmetric and $-1$ being antisymmetric. This then determines whether the spin states can be symmetric or antisymmetric, and which $S$ are allowed. 

The term symbol is then:
\[ ^{2S+1}L_J\]

Note: AFAIK just consider the unfilled orbitals. 

\subsection{Hund's Rule}
\begin{enumerate}
    \item Largest value of $S$ is lowest in energy: Aligned spins mean they are farthest apart from each other.
    \item Following 1, largest $L$ is lowest in energy: Electrons with same angular momentum or orbiting in the same direction, and hence encounter each other the least and therefore further apart on average.
\end{enumerate}


\subsection{Spin Orbit Interaction}
Because of relativistic \textit{spin-orbit} interactions (whatever that means), the hamiltonian is altered to be:
\[ \hat{H} = \hat{H} + \hat{H}_{SO}\]
This extra term:
\[ \hat{H}_{SO} \propto L \cdot S \]

There are two types of magnetic dipole moment associated with an electron:
\begin{itemize}
    \item One related to the orbital motion of the electron
    \item One related to the spin of the electron
\end{itemize}
Hence, spin-orbit interaction. These give rise to \textit{fine structure}.

\subsubsection{Orbital magnetic moment}
This is given by:
\[ \vec{\mu}_l = -\frac{e}{2m_e}\vec{l} = \gamma_e \vec{l}\]
For example:
\[\mu_{(l_z)} = \gamma_e l_z = -\frac{e}{2m_e}m_l\hbar = -\mu_bm_l\]

Where the Bohr Magnetron is given by: \(\mu_B = \frac{e\hbar}{2m_e}\)

\subsubsection{Spin magnetic moment}
A classical approximation is given by:
\[ \vec{\mu}_s = -\frac{e}{2m_e}\vec{s}\]
The Dirac equation appends a factor of $2$ to that, but then experiments on an electron gives a factor of:
\[g_e = 2.002319304\]
Appending this factor gives:
\[ \vec{\mu}_{(s_z)} = -g_e \mu_B m_s \]
And for an electron, $m_s =\mp \frac12$.

\subsubsection{Interaction}
Moving in an E field is like moving in a B field, except:
\[ \vec{B} = \frac{\vec{E}\times \vec{v}}{c^2}\]
Let the electric field be the gradient of an isotropic electric potential $\psi(r)$:
\[ \vec{E} = -\frac{\vec{r}}{r}\frac{d\psi}{dr}\]
Then doing some substitutions into the first equation above, and noting momentum being a cross product of position and momentum, then using the interaction energy of the magnetic dipole:
\[ V = -\vec{\mu} \cdot \vec{B}\]

This gives:
\[H_{SO} = -\vec{\mu_s}\cdot\vec{B} = -\frac{e}{m^2_ec^2 r}\frac{d\phi}{dr}{\vec{l}\cdot \vec{s}}\]

Which is twice the result from the relativistic Dirac equation. For a one-electron atom:

\[\phi = -\frac{Ze}{4\pi\epsilon_0 r}\]

Which means:
\[H_{SO} \propto \frac{Z}{r^3}\]

I have no idea what is going on here, but:

\[ \langle \frac{1}{r^3} \rangle = \langle \Psi_{nlm} | \frac1{r^3} |\Psi_{nlm}\rangle = \frac{Z^3}{n^3a^3_Ml(l+\frac12)(l+1)}\]
So
\[H_{SO} \propto \frac{Z^4}{n^3}\]

\subsection{Spin Orbit Operator}
The spin orbit Hamiltonian derived before does not have a fixed projection on the z axis. However, $J=L+S$ does. This gives:

\[H_{SO} = hc A(L,S) \vec{L} \cdot {S}\]
Where:
\[A(L,S) = -\frac{1}{hc}\frac{e}{m^2_ec^2 r}\frac{d\phi}{dr}\]

As $J^2 = (L+S)^2 = L^2+S^2+2L\cdot J$, $\vec{L}\cdot\vec{S}$ can be rewritten as:
\[ \vec{L}\cdot\vec{S} = \frac12 [\vec{J}^2 - \vec{L}^2 - \vec{S}^2]\]

By first order perturbation theory (see QM notes), the (first order) energy shift is:

\[\frac{\Delta E_{SO}}{hc} = \frac{A(L,S)}{2}[J(J+1) - L(L+!) - S(S+1)]\]

Which the Term symbol now makes sense:
\[ ^{2S+1}L_J\]

The \textit{Lande Interval Rule} is known as:

\[\Delta E_{SO}(J) - \Delta E_{SO}(J-1) = hc A(L,S) J\]

\subsection{Coupling Schemes}
\subsubsection{$LS$ coupling (Russell-Sanders)}

For atoms with low nuclear charge (low Z), spin orbit interaction is weaker than the electron-electron interactions.

The most appropriate way to describe the system is to consider $\vec{S}, \vec{L}, \vec{J} = \vec{L}+\vec{S}$.  

Hund's Third Rule:
\begin{enumerate}[(a)]
    \item Normal Case - Less than half full: Level energies increase with increasing $J$. $\therefore$ Smallest $J$, lowest energy.
    \item Inverted Case - More than half full: Level energies decrease with increasing $J$.
    \item If the outer shell is half full, there is no multiplet energy splitting.
\end{enumerate}


\subsubsection{$jj$-coupling}

In high Z atoms, the coupling constant depends strongly on Z:
\[A(L,S) \propto Z^4\]
Spin-orbit interaction for each electron is strong, so each $\vec{j}_i = \vec{l}_i +\vec{j}_i$ can be computed for each electron and the sum would give $J$.

Then:
\[J = |j_1-j_2|,...,|j_1+j_2| \quad \text{in steps of 1}\] 

\subsection{Parity in many-electron atoms}

The parity is a product of the parity of individual atoms:

\[(-1)^{l_1}(-1)^{l_2}(-1)^{l_3}\times...\times(-1)^{l_n} = (-1)^{\sum_i l_i}\]

For the term symbol, odd parity has a $o$ superscript while even parity has nothing:
\[ ^{2S+1}L_J^o\]

\subsection{Hyperfine structure}

Protons are also fermions with spin angular momenta $frac12$. Thus there can also be a magnetic dipole moment associated with protons:

\[\vec{\mu}_I = g_N \left(\frac{m_e}{m_p}\mu_B\right)\frac{\vec{I}}{\hbar} = g_N \mu_N \frac{\vec{I}}{\hbar}\]

The energy level shifts are analogous to spin orbit interactions, and are dependent on the interaction of the nuclear magnetic moment with total electron angular momentum and total electron spin magnetic moment:

\[\Delta E_{HFS} \propto \vec{I} \cdot \vec{J} \text{ and} \propto \vec{I} \cdot \vec{S}\]
The energy shifts are known as hyperfine structure. These lift the degenerate energy levels by $F$, given by:
\[\vec{F} = \vec{J} + \vec{I}\]
Again, they take values of:
\[F = |J-I|,...,|J+I| \text{ in steps of 1}\]

\section{Atomic Spectra}

\subsection{Fermi's Golden Rule}
The transition probability between an initial and final state can be given by:
\[T_{if} = |\langle \Psi_f | V_{dip}|\Psi_i\rangle|^2\]

Page 84 onwards gives an example of an electron in an oscillating electric field. 

\subsection{Selection Rules revisited}
\begin{description}
    \item [Rigorous Rules] These rules must be followed
    \begin{itemize}
        \item $\Delta J = 0, \pm 1$ but $J=0 \not\to J' = 0$
        \item $\Delta M_J=0,\pm1$
        \item Parity must change
    \end{itemize}
    \item [Strong Transitions] These rules, if not followed, give rise to weak transitions
    \begin{itemize}
        \item  $\Delta S = 0$
        \item $\Delta L = 0, \pm 1$ but $L=0 \not \to L' = 0$
    \end{itemize}
\end{description}
See page 87 for example. 

\subsection{Photon emission and absorption}
Consider a two state system with energy and number of atoms $(E_1,N_1)$ and $(E_2,N_2)$. The population can be expressed as:

\begin{align*}
    \frac{dN_1}{dt} = -C\Upsilon(\upsilon_{12})N_1 + AN_2 + B\Upsilon(\upsilon_{12})N_2
    \frac{dN_2}{dt} = +C\Upsilon(\upsilon_{21})N_1 - AN_2 - B\Upsilon(\upsilon_{12})N_2
\end{align*}

Equating both to zero, and expressing in terms of the Maxwell-Boltzmann distribution:

\[A = \frac{8\pi h \upsilon^3_{12}}{c^3}B\]
And:
\[B = \frac{2\pi}{3\hbar^2}\left(\frac{1}{4\pi\epsilon_0}\right)\left|\int\Psi^*_2\vec{\mu}\Psi_1d\tau\right|^2\]

Where:
\begin{align*}
    A &\equiv \text{Einstein A coefficient for spontaneous emission}\\
    B &\equiv \text{Einstein B coefficient for stimulated emission}
\end{align*}

\section{Zeeman Effect}

\subsection{Normal Zeeman}
Total Spin $S = 0$. Therefore related to the magnetic moment generated from the orbital angular momentum $L$. Split into $2L+1$ energy levels. Transition rules: $\Delta M_L = 0, \pm 1$.

\subsection{Anomalous Zeeman}
This gives rise to additional splitting beyond the split arising from spin-orbit interaction.

When $S\neq 0$. Unfilled orbitals, unpaired electrons. Spin-orbit interactions occur, and the vector $J=L+S$ precesses about the Lamour frequency under $B$ field.

The Zeeman splitting is smaller than that from spin orbit interaction. THis leads to transitions of $\Delta M_J = 0, \pm 1$.

The Lande g-factor is given by:
\[ g_J = 1 + \left[\frac{J(J+1)+S(S+1)-L(L+1)}{2J(J+1)}\right]\]


\section{Stark Effect}
Pertubation given by:
\[ V_F = eF_zz\]

\subsection{In Weak Fields}
Gives rise to shifts and splittings:
\begin{itemize}
    \item Degenerate energy levels with different orbital angular momentum quantum number $\implies$ Linear Stark Shifts
    \item Non degenerate $\implies$ quadratic Stark Shifts
\end{itemize}

\subsubsection{Quadratic Stark Effect}
No $l$ degeneracy means no electric dipole moment means no Stark shifts. However, the E field induces an electric dipole. Using second order Perturbation theory, the transition in the ground state of H:

\[\Delta E_F = e^2 F_z^2 \sum_{n'}\frac{|\int \si^*_{n10}\hat{z}\psi_{100}d\tau|^2}{E_{100}-E_{n'10}}=-\frac12\alpha F^2_z\]

The induced electric dipole moment can be obtained by $dE_F/dF_z=\alpha F_z$, so:
\[\alpha = 7.42\times 10^{-41}Cm^2V^{-1}\]

The quadratic effect lowers the energy of an isolated non-degenerate energy level. This is due to the induced electric dipole moment being oriented parallel to the field vector.

\subsubsection{Linear Stark Effect}

Occurs when atom possesses an intrinsic electric dipole moment. This is when the electron is in an orbital with $l$ (orbital angular momentum number) that is degenerate with other orbitals (such that the combination of spherical harmonics do not equal zero). The eigenstates take the form:

\begin{align*}
    \psi_1 &= \psi_+= \frac{1}{\sqrt{2}}(\psi_{200}+\psi_{210})\\
    \psi_2 &= \psi_- =\frac{1}{\sqrt{2}}(\psi_{200}+\psi_{210})\\
    \psi_3 &= \psi_{2,1,-1}\\
    \psi_4 &= \psi_{2,1,+1}
\end{align*}

\paragraph{Energy Shifts}
The field does not induce a dipole moment, so the energy shift is:
\[ \Delta E_F \propto F_z\]

\section{Born-Oppenheimer Approximation}

Their insight was that electronic wavefunctions are insensitive to the changes in positions of the nuclei. After a lot of math:

\[\left[-\frac{\hbar^2}{2\mu_{AB}}\nabla^2_R + E_{\text{elec}}\right]v(\vec{R}) = Ev(\vec{R})\]
The electron energies act as an effective potential to the movement of the nuclei.

\subsection{Hydrogen Ion}
The $H_2^+$ ion has one electron and two protons. In atomic units, the Hamiltonian is given by the electron's KE and three terms of $E_{\text{elec}}$:
\[\hat{H}_{\text{elec}} = -\frac{\nabla_r^2}{2}-\frac{1}{r_A}-\frac{1}{r_B}+\frac{1}{R}\]

\subsection{LCAO}
When the electron is close to either of the proton, the wavefunction reduces to the 1s orbital. Thus, the total wavefunction has to be a superposition of the two. This method is known as LCAO, or Linear Combination of Atomic Orbitals.

\[\psi = C_1\phi_{1s}(r_A) + C_2\phi_{1s}(r_B) \]

$H^2_+$ is a homonuclear diatomic molecule, so the probability of finding the electron at either proton has to be the same (symmetry). Thus, in the superposition, the coefficient squared has to be equal. 

\[|C_1|^2 = |C_2|^2 = |C|^2\]

In the superposition: If add = symmetric = gerade. If minus = antisymmetric = ungerade.

Normalising to their respective coordinates $dr_A, dr_B$:

\begin{align*}
    \langle \psi | \psi \rangle &= 1\\
    \therefore C &= \frac{1}{\sqrt{2(1\pm I(R)}}\\
    \text{where } I(R) &= \langle \phi(r_A)|\phi(r_B)\rangle = \langle \phi(r_B)|\phi(r_A)\rangle  
\end{align*}

See page 112 onwards for a further discussion. 

\subsection{Hydrogen molecule}
The Hamiltonian is given by the two electron's KE and the potential energies due to the repulsion and attraction:
\[\hat{H}_\text{elec} = -\frac{\nabla^2_1}{2} - \frac{\nabla_2^2}{2} - \frac{1}{r_{A1}} - \frac{1}{r_{A2}} - \frac{1}{r_{B1}} - \frac{1}{r_{B2}} + \frac{1}{r_{12}} + \frac{1}{R}\]

The full wavefunction is:
\[\psi^{S,T} = \psi(\vec{R};\vec{r}_i)\chi^{S,T}\]
With S,T referring to the singlet or triplet states.

Since there are two electrons, if the spatial part is symmetric, the spin part must be antisymmetric. The mathematical treatment is done from page 122 to 126. 

The expression giving the energy is:
\[E_\pm = 2E_{1s} + \frac1R + \frac{J(R)}{1\pm I(R)^2} \pm \frac{K(R)}{1\pm I(R)^2}\] 
Here, I(R) is the overlap integral, K(R) is the Exchange integral and J(R) is the Colomb Integral.

In summary:
\begin{itemize}
    \item Singlet state has energy $E_+$. Shifted to lower energy.
    \item Triplet state has energy $E_-$. Shifted to higher energy. 
\end{itemize}
The singlet state has a symmetric spatial wavefunction, leading to a nonzero probability of finding the electrons between the two nuclei. This leads to a minimum of electronic energy, forming a stable molecule. 

\subsubsection{Molecular Orbitals}
The equivalent to the 1s, 2s, 2p orbitals in atoms, the molecules have:

\begin{align*}
    1\sigma_g &= \psi_+ = \frac{1}{\sqrt{2}}[\phi_{1s}(\vec{r}_a)+\phi_{1s}(\vec{r}_B)]\\
    1\sigma_u &= \psi_+ = \frac{1}{\sqrt{2}}[\phi_{1s}(\vec{r}_a)-\phi_{1s}(\vec{r}_B)]\\
\end{align*}
Where `g' refers to gerade and `u' refers to ungerade.

The ground state $\psi^S$ (bonding) is:
\[1\sigma_g^2\]
Whereas the excited state $\psi^T$ (antibonding) is:
\[1\sigma_g1\sigma_u\]

\subsection{Heteronuclear Molecule}
No gerade or ungerade, constants are not equal, leads to hybrid orbitals.

\section{Nuclear Motion}
N nuclei lead to 3N nuclear degrees of freedom:
\begin{enumerate}
    \item Translational: 3 degrees in 3D
    \item Vibrational: $3N-6$, unless linear (diamotic), then $3N-5$.
    \item Rotational: About its COM. 3 except for linear molecules, which only has 2 degrees of freedom (only 2 axis).
\end{enumerate}


The Kinetic energy part of the Hamiltonian can be decomposed in spherical coordinates. 

\subsubsection{Rotation}
The rotational energy is given as:

\[E_\text{rot} = \frac{\hbar^2}{2\mu_{AB}R^2} J(J+1)\]

Rigid rotor assumption: Internuclear distances stay constant, unaffected by rotation. Its moment of inertia becomes:

\[ I = M_A(R-x)^2+M_Bx^2\]
Where the masses are the respective masses of the two nuclei, and R is the internuclear distance. 

If $R=R_e$, then in the COM frame of reference:
\[I_e = \mu_{AB}R^2_e\]
So:
\[E_\text{rot} = \frac{\hbar^2}{2I_e}J(J+1) = BJ(J+1)\]

The rotational constant is given by $B$.

Divide by $hc$ to get the constant in terms of $cm^{-1}$:
\[ B = \frac{h}{8\pi^2c\mu_{AB}R^2_e}\]

Thus:
\[ E_\text{rot} \propto \frac{1}{\mu_{AB}} \propto \omega_\text{rot}\]

\subsubsection{Vibration}
The term related to vibrational energies can be approximated (by Taylor expansion) to be a quantum harmonic oscillator around $R_e$. 

With:
\[k = \left.\frac{d^2V}{dx^2}\right|R=R_e\]
The SE becomes:
\[\left[ -\frac{\hbar^2}{2\mu_{AB}}\frac{\partial^2}{\partial R^2} + \frac12 k(R-R_e)^2 + V(R_e)\right] F(R) = E_\text{vib}F(R)\]

The energies are given as:
\[E_v = \hbar \omega_0(v+\frac12)\]
Where $v$ is the vibrational quantum number. The fundamental vibrational frequency is:
\[\omega_0 = \sqrt{\frac{k}{\mu_{AB}}}\]
Thus the zero point energy of the oscillator is:
\[E_{v=0} = \frac12\hbar\sqrt{\frac{k}{\mu_{AB}}} \neq 0\]


\subsection{Total Energy of Diatomic Molecule}

When:
\begin{enumerate}
    \item $v$ is small (near bottom of well so harmonic approximation is reasonable)
    \item $J$ is small (centrifugal distortion is minimal, i.e. the molecule is not rotating rapidly causing the bond length to increase
\end{enumerate}
\[ E_{AB} = V(R_e) + BJ(J+1) + \hbar\sqrt{\frac{k}{\mu_{AB}}}(v+\frac12)\]

\subsubsection{Corrections to the Energy}

\paragraph{Rotational Energy}
The rotational energies can be adjusted by considering:

\begin{enumerate}[i]
    \item Adjust the rotational constant at high $v$ to be:
    \[B_v = \langle|\frac{\hbar^2}{2\mu_{AB}R^2}|\rangle\]
    \item The centrifugal distortion at high $J$ can be adjusted by introducing a second order correction, $E_J^(2)$:
    \[ E_J^(2) = -D_vJ^2(J+1)^2\]
    Where $D_v$ is the \textit{centrifugal distortion constant} and is much smaller than $B_v$ typically.
\end{enumerate}

This leads to:
\[E_J = B_vJ(J+1) - D_vJ^2(J+1)^2\]

\paragraph{Vibrational Energies}
The vibrational energies can be adjusted by replacing the harmonic potential by Morse's potential at higher $v$:
\[ V_\text{Morse}(R) = -D_e + D_e[1-e^{-\alpha(R-R_e)}]^2\]

Expanding in terms of $(R-R_e)$:
\[ \alpha = \sqrt{\frac{k}{2D_e}}\]

The energy of each state in Morse potential can be expressed as:

\[E_v = \hbar \omega_0(v+\frac12) - \frac{[\hbar\omega_0(v+\frac12)]}{4D_e}\]
Where
\[\omega_0 = \alpha \sqrt{\frac{2D_e}{\mu_{AB}}}\]

\section{Molecular Spectra}

\subsection{Spectra}
Molecular spectra are composed of three main kinds of transitions:
\subsubsection{Pure Rotational Transitions}
    \begin{itemize}
        \item Does not involve change in electronic state or vibrational state
        \item Strong for molecules with non-zero electric dipole moments an oscillating EM field can interact with (i.e. LiF, HCl)
        \item Homonuclear diatomic molecules do not undergo pure rotational transitions within the electric dipole approximation (i.e. $H_2$, $D_2$, $H_2^+$)
        \item Transitions occur at frequences on the order of rotational energy spacings, which is the microwave region
    \end{itemize}
    
By Fermi's Golden Rule:

\[T_{if} = [\langle \psi_f|V_\text{dip}|\psi_i\rangle]^2\]
Where $V_\text{dip} = -\vec{\mu}_\text{elec} \cdot \vec{F}$

Rewriting in terms of spherical harmoincs, it is determined that: $\Delta J = J' - J = \pm 1$ and $\Delta M_J = 0$. Thus:
\begin{align*}
    \Delta E_J &= E_{J+1} -E_J\\
    &= B(J+1)(J+2) - BJ(J+1)
    &= 2B(J+1)
\end{align*}

Thus the spacing between each J increases by $2B$ (previous energy spacing $+2B$).

\subsubsection{Vibrational Rotational Transitions}

\begin{itemize}
    \item Do not involve change in electronic state
    \item Involves changes in the rotational and vibrational states
    \item Same order of magnitude as $\omega_0$, lies in Infrared region
    \item \[ \Delta v = \pm 1\]
    \item \[ \Delta J = \pm 1\]
\end{itemize}

\paragraph{P branch and R branch}
The rotation vibration spectra exhibit two groupings of lines:
\begin{itemize}
    \item $\Delta J = -1$ is the P branch:
    \[\Delta E_\text{ro-vib} = E(v+1,J-1) - E(v,J) = -2BJ+\hbar\omega_0\]
    \item $\Delta J = +1$ is the R branch:
    \[\Delta E_\text{ro-vib} = E(v+1,J+1)-E(v,J) = 2B(J+1)+\hbar\omega_0\]
\end{itemize}

\subsubsection{Electronic Transitions}
Diatomic molecules have axial symmetry, i.e. an axis which can be taken as the $z$ axis which would have a project of momentum $L_z$ which is constant. Let the quantum number from the eigenvalue of $L_z$ be $\Lambda$. The term symbol is given as:
\[ ^{2S+1}\Lambda^{+,-}_{g,u}\]
Where $+,-$ denotes reflection symmetry and $g,u$ represents the orbital symmetry.

Similar to S,P,D,F,G,...:
\begin{table}[h]
    \centering
    \begin{tabular}{c||c|c|c|c|c}
         $|\Lambda|=$& 0 & 1 & 2 & 3 & ... \\
         \hline\hline
         & $\Sigma$ & $\Pi$ & $\Delta$ & $\Phi$ & ...
    \end{tabular}
    \caption{Molecular term symbols}
    \label{tab:molets}
      \begin{tabular}{c||c|c|c|c|c}
         $\lambda$& 0 & 1 & 2 & 3 & ... \\
         \hline\hline
         & $\sigma$ & $\pi$ & $\delta$ & $\phi$ & ...
    \end{tabular}
    \caption{One-electron Molecular Orbitals Molecular term symbols}
    \label{tab:onemolets}
\end{table}

\subsubsection{Selection Rules}
Occurs in ultraviolet spectrum:
\begin{align*}
    \Delta \Lambda &= 0, \pm 1\\
    \Delta S &= 0\\
    \text{Symmetry must change } g\to u &\text{ or } u \to g\\
    + \to + &\text{ or } - \to -\\
    \Delta v &= \text{any value}\\
    \Delta J &= 0, \pm 1
\end{align*}

\subsection{Frank-Condon Principle}
From Fermi's Golden Rule:
\[ T_{if} \propto \left[ \langle \psi_f|\vec{r}|\psi_i\rangle \right]^2\]

For a diatomic molecule with wavefunction:
\[\Psi(R;r) = v(R)\psi(R;r)\]
By considering only the vibrational part $F(R)$, the transition probability is:
\[T_{if} = \left[ \langle \psi_f|\vec{r}|\psi_i\rangle \right]^2 \times \underset{\text{Frank-Condon Factor}}{[\langle F_f(R)|F_i(R)\rangle ]^2}\]

Where this is known as the Frank Condon overlap integral:
\[[\langle F_f(R)|F_i(R)\rangle ]^2 = \int F_f^*(R) F_i^*(R) dR\]

The band with the highest transition probability is the one for which the
transition is “vertical”. The molecule finds itself in the excited electronic state
with the same inter-nuclear separation as it had in the ground electronic state.

The most probable excitation is in the classical turning point.
\end{document}