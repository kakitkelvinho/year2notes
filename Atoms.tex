\documentclass[12pt]{article}
\usepackage[utf8]{inputenc}
\usepackage{amsmath}
\usepackage{amssymb}
\usepackage[a4paper, total={6.5in, 9.5in}]{geometry}



\title{Atom and Molecular Physics}
\author{Kelvin Ho}
\date{April 2021}

\begin{document}

\maketitle

\section{Introduction}
This is a continuation of the first Quantum Mechanics course.

\section{One-electron Atoms}

\subsection{Hydrogen}

Solving the Schrodinger Equation for the Hydrogen Atom is to write the potential of the system as the Colomb potential between the electron and the ion core:

\[V(r) = \frac{-Ze^2}{4\pi\epsilon_0r}\]

Then, the energy levels are:

\[ E_n = -\frac{m_eZ^2e^4}{32\pi^2\epsilon_0^2\hbar^2n^2}\]

The energy emitted when an electron jumps is the change between the energies of the system:

\[\Delta E_n = -\frac{m_eZ^2e^4}{32\pi^2\epsilon_0^2\hbar^2}\left(\frac{1}{n_1^2} - \frac{1}{n_2^2}\right)\]

Rydberg empirically discovered that:

\[\frac{1}{\lambda_{if}} = R\left(\frac{1}{n_f^2} - \frac{1}{n_i^2}\right)\]

Compared with the first equation and Planck's formula $E = \frac{hc}{\lambda}$:

\[ R = \frac{m_eZ^2e^4}{64\pi^2\epsilon_0^2\hbar^3c}\]


\subsection{Finite Mass Core}
The reduced mass is:

\[\mu = \frac{1}{m_p} + \frac{1}{m_e} = \frac{m_e m_p}{m_e + m_p}\]

For example, for the adjusted Rydberg constant of Hydrogen:

\[ R_H = R_\infty \frac{\mu_H}{m_e}\]

In general, for a one electron atom of mass $M$ and ion core charge given by $Ze$:

\[R_M = R_\infty\frac{\mu_M}{m_e}Z^2\]

\subsection{Wavefunction}
The wavefunction is solved in the previous quantum module. They can be represented as:
\[ \psi = R(r)Y^m_l(\theta,\phi)\]

\subsubsection{Angular Part}
The nature of the wavefunction indicates that the electron's angular momentum is quantised and relies on $l$ and $m$, which is the \textit{orbital angular momentum quantum number} and the \textit{magnetic/azimuthal quantum number}. They can vary in integer levels:
\begin{align*}
    l &= 0, 1, 2,..., n - 1\\
    m &= -l, -l+1,...,l-1,l
\end{align*}

The $l$ values are also labelled with alphabets $s,p,d,f,g,h,...$.

\paragraph{Parity}
This is defined to be:
\[\hat{P} f(\vec{r}) \equiv f(-\vec{r})\]
Even parity:
\[\hat{P} f(\vec{r}) \equiv f(-\vec{r}) =  f(\vec{r})\]
Odd Parity:
\[\hat{P} f(\vec{r}) \equiv f(-\vec{r}) =  -f(\vec{r})\]

When applied to spherical harmonic function, this is multiplying the wavefunction with $(-1)^m$.

\subsubsection{Radial Part}

Solving for the radial wavefunction indicates that this only depends on $n$ and $l$.

\subsubsection{Degeneracy}

The energy level, shown by the Rydberg formula, is only dependent on the principle quantum number $n$. Given that there are $n-1$ $l$ for each $n$ and $2l+1$ $m$ for each $l$, there are a total of:

\[\sum^{n-1}_{l=0} (2l+1) = n^2\] degenerate states.

Reasons for degeneracy:$\frac1r$ Potential is only dependent on magnitude of $r$ but not on the (angular) direction. 

\subsection{Spin}

Arises from the relativistic treatment of electron motion. Is very similar to the analysis of angular momentum. Instead of $l$ and $m$, there are $s$ and $m_s$, representing the spin vector and the projection of the spin vector in the $z$ dimension. The full wavefunction is adjusted by introducing the spin dependence:

\[\psi = R_n(r)Y^l_m(\theta, \psi) \chi^s_{m_s}\]

\subsection{Transitions}

When a photon is absorbed, the electron `jumps'. The selection rules are:

\begin{itemize}
    \item $\Delta n$: Any value
    \item $\Delta l = \pm 1$: TO conserve angular momentum
    \item $\Delta m = 0, \pm 1$
\end{itemize}    
    
    
\section{Many Electron Atom}

This is more difficult to solve because the potential would hence be dependent upon the interaction between the electrons. 

\subsection{Independent Particle Model}
First ignore the electron electron repulsion. Solve for each and every electron and solve for it. Then correct for it.

\subsection{Central Field Approximation}
Add another potential term:
\[ V_c(r_i) = \langle \frac{e^2}{4\pi\epsilon_0 r_{ij}}\rangle_{j\neq i}\]
Which is basically the Colomb potential between different electrons. Then, solve $N$ one-electron Schrodinger equation, and the total energy is the sum of the energy from each of the solutions. 

The potential is not dependent on $\frac{1}{r}$ anymore.

This method is best applied to atoms or ions with a single electron occupying an outer orbital:
\begin{itemize}
    \item Alkali Metal Atoms in Group I
    \item Slightly Ionised Alkaline Earth Metals in Group II (After losing one electron in the S orbital)
    \item Rydberg atoms: These have one electron in the highly excited state (high value of n)
\end{itemize}

\paragraph{Screening}
The inner electrons \textit{screen} the charge of the nuclues:

When $r_i \to 0$:
\[ -\frac{Ze^2}{4\pi\epsilon_0 r_i} + V_c(r_i) \implies -\frac{Ze^2}{4\pi\epsilon_0 r_i}\]

When $r_i \to \infty$:
\[ -\frac{Ze^2}{4\pi\epsilon_0 r_i} + V_c(r_i) \implies -\frac{e^2}{4\pi\epsilon_0 r_i}\]

\subsection{Quantum Defects}

Solving the Schrodinger Equation leads to an adjustment to the eigenenergies:

\[\frac{E_{nl}}{hc} = -\frac{R_M}{(n-\Delta_{nl})^2}\]

Where $\Delta_{nl}$ is the quantum defect and $(n-\Delta_{nl})$ is the effective principal quantum number. The adjustment to the Rydberg constant uses a $Z$ that is the effective combined charge of the nucleus and the core electrons. See page 43 for a table listing of some quantum defect values.

\subsection{Indistinguishable particles}

If two particles are indistinguishable by measurement, then:

\[ |\psi(1,2)|^2 = |\psi(2,1)|^2\]

Which means:
\[ \psi(1,2) = +\psi(2,1)\]
Which is known as symmetric w.r.t. exchange.

Or:
\[ \psi(1,2) = -\psi(2,1)\]
Which is known as antisymmetric w.r.t. exchange.

\subsection{Pauli Principle}

\begin{itemize}
    \item Fermions: Antisymmetric
    \item Bosons: Symmetric
\end{itemize}


\end{document}